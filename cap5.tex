\chapter{Conclusions}
We developed a simple game just using a slightly modified version of a well known algorithm such as Alpha-Beta Pruning, and tested it to find out whether it was challenging or not.
It was, and quite funny.

We presented the game to a reasonable number of human players, and the very most of the times the player received a very ugly defeat.

We also developed a branch of the game in which two AI moved planes fight against each other.
These experiments confirmed our hipothesis that incrementing the search depth does not always increase the outcome, in fact in more than one matches a AI player with a search depth value of 6 beated another AI player who had a search depth value of 10.

As further developments, we wondered about:
\begin{itemize}
\item Different types of plane, each with its own moves, weapons and behaviour.
\item Bonuses and maluses on the map.
\item Increasing the number of possible players
\item Increasing the number of cards
\item Transforming the field of play from a plane to a manifold
\item Adding one dimention to the game (plane altitude)
\end{itemize}

Many of these improvements would certainly increase the branching factor of the search, hence it is a very good thing to be able to make good decisions without the need of big search depth values.