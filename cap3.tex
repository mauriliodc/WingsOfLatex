
\chapter{Heuristics}

\section{State representation}

The first step toward an heuristic function for \emph{Wings of War} game is to find a 
suitable \textbf{state representation} for it. The only elements in the games are the two
players planes. For this reason the only useful data that can be used to build an heuristic
function are closed inside the planes parameters.

Each plane has

\begin{itemize}
  \item \textbf{Life}, the amount of \emph{healt points} for the plane. When $life=0$ the
    plane is destroyed.
  \item \textbf{Position}, the coordinates $<x,y>$ of the plane center.
  \item \textbf{Orientation}, the angle $\theta$ of the plane respect to the x axis.
\end{itemize}

A state $s$ of the game is completely represented by these parameters. Togheter with the
other \textbf{fixed parameters} like range, field of view and world size they make a
complete game description.

\section{Wings of War heuristic function}

The goal of the heuristic function is to quantify \emph{how good} is a state for one
player. Heuristic function has to answer the question: how much the state $s$ is
desiderable for player $x$?

This answer is the key for the informated search algorithm described in the next chapter.

A general heuristic function has the form of equation (\ref{eq:heuristic})

\begin{equation}
  H(s) = Score_{AI}(s) - Score_{Player}(s)
  \label{eq:heuristic}
\end{equation}

The heuristic function $H(s)$ map states $s \in \Sigma$ to $\mathbb{Z}$. $H(s)$ has
positive values in the more desiderable states for the AI and negative values otherwise.
In the extreme cases $H(s^\star)=+\infty$ means that the state $\star{s}$ is a winning state
for the AI and $H(\star{s})=-\infty$ means that the state $\star{s}$ is a wining state for
the player.

The question now is: how can we compute the function $Score_X(s)$? In general a score
function has the form

\begin{equation}
  Score_X(s) = \sum_{i=1}^N w_i \cdot f_i
\end{equation}


\begin{equation}
  Score_X(s) = w_1 \cdot ShotValue(X,s) + w_2 \cdot RemainingLife(X,s)
  \label{eq:heuristicscore}
\end{equation}


